\isection{Описание алгоритма}

    Рассмотрим расширенную прямоугольную матрицу состоящую из коэффициентов:
    \begin{equation}
        M = \left( \begin{array}{cccccc|c}
           a_{11} & a_{12} & \cdots & a_{1 q} & \cdots & a_{1 n} & b_1 \\ 
           a_{21} & a_{22} & \cdots & a_{2 q} & \cdots & a_{2 n} & b_2 \\ 
           \vdots & \vdots & \ddots & \vdots & \ddots & \vdots & \vdots \\ 
           a_{p 1} & a_{p 2} & \cdots & a_{pq} & \cdots & a_{pn} & b_{p} \\ 
           \vdots & \vdots & \ddots & \vdots & \ddots & \vdots & \vdots \\ 
           a_{n 1} & a_{n 2} & \cdots & a_{nq} & \cdots & a_{nn} & b_{n}
       \end{array} \right).
    \label{eq:matrix_m}\end{equation} 

    Выберем наибольший по модулю элемент \( a_{pq} \), не принадлежащий столбцу свободных членов матрицы \( M \). Этот элемент называется \textit{главным элементом}. Строка и столбец матрицы \( M \), содержащие главный элемент, называются \textit{главной строкой} и \textit{главным столбцом} соответственно.

    Пусть \( i = \overline{1,n}, ~ i \neq p \), из каждой \( i \)-й строки матрицы \( M \) вычтем \( p \)-ю строку умноженную на \( - \rfrac{a_{iq}}{a_{jq}} \). Очевидно, что после этого в \( q \)-столбце все элементы будут равны нулю кроме элемента \( a_{pq} \). Назовем полученную матрицу \( M_* \). Скажем, что \( M^{(1)} \) -- матрица \( M_* \) после удаления из нее главной строки и главного столбца.

    Проделываем эти же действия с матрицей \( M^{(1)} \), получаем матрицу \( M^{(2)} \). Повторяем алгоритм, пока не получим матрицу \( M^{(n - 1)} \), которая из себя представляет двух элементную матрицу -- строку, которая также является главной.

    Для получения системы с треугольной матрицей, эквивалентной изначальной системе, объединяем все главные строки матриц \( M, ~ M^{(1)}, ~ M^{(2)}, ~ \dots,\) \( M^{(n - 1)} \), начиная с последней \( M^{(n - 1)} \).

    Решить получившуюся систему можно, последовательно идя по системе находя значения новых неизвестных, как примере с следующей треугольной системой:
    \[
        \begin{cases}
            x_1 = \beta_1; \\ 
            x_2 + \alpha_{21} x_1 = \beta_2; \\ 
            x_3 + \alpha_{32} x_2 + \alpha_{31} x_1 = \beta_3; \\ 
            x_4 + \alpha_{43} x_3 + \alpha_{42} x_2 + \alpha_{41} x_1 = \beta_4.
        \end{cases}
    \] 

    Решение:
    \[ \begin{matrix}
        x_1 & = & \beta_1, \\ 
        x_2 & = & \beta_2 - \alpha_{21} x_1, \\ 
        x_3 & = & \beta_3 - \alpha_{32} x_2 - \alpha_{31} x_1, \\ 
        x_4 & = & \beta_4 - \alpha_{43} x_3 - \alpha_{42} x_2 - \alpha_{41} x_1.
    \end{matrix} \]
